% Этот шаблон документа разработан в 2014 году
% Данилом Фёдоровых (danil@fedorovykh.ru) 

\documentclass[t, aspectratio=43]{beamer}
\usepackage[]{graphicx}
\usepackage[]{color}
%% maxwidth is the original width if it is less than linewidth
%% otherwise use linewidth (to make sure the graphics do not exceed the margin)
\makeatletter
\def\maxwidth{ %
	\ifdim\Gin@nat@width>\linewidth
	\linewidth
	\else
	\Gin@nat@width
	\fi
}
\makeatother

\definecolor{fgcolor}{rgb}{0.345, 0.345, 0.345}
\newcommand{\hlnum}[1]{\textcolor[rgb]{0.686,0.059,0.569}{#1}}%
\newcommand{\hlstr}[1]{\textcolor[rgb]{0.192,0.494,0.8}{#1}}%
\newcommand{\hlcom}[1]{\textcolor[rgb]{0.678,0.584,0.686}{\textit{#1}}}%
\newcommand{\hlopt}[1]{\textcolor[rgb]{0,0,0}{#1}}%
\newcommand{\hlstd}[1]{\textcolor[rgb]{0.345,0.345,0.345}{#1}}%
\newcommand{\hlkwa}[1]{\textcolor[rgb]{0.161,0.373,0.58}{\textbf{#1}}}%
\newcommand{\hlkwb}[1]{\textcolor[rgb]{0.69,0.353,0.396}{#1}}%
\newcommand{\hlkwc}[1]{\textcolor[rgb]{0.333,0.667,0.333}{#1}}%
\newcommand{\hlkwd}[1]{\textcolor[rgb]{0.737,0.353,0.396}{\textbf{#1}}}%

%\AtBeginSection[]
%{
%	\begin{frame}<beamer>
%		\frametitle{Outline for section \thesection}
%		\tableofcontents[currentsection]
%	\end{frame}
%}

\usepackage{framed}


\definecolor{shadecolor}{rgb}{.97, .97, .97}
\definecolor{messagecolor}{rgb}{0, 0, 0}
\definecolor{warningcolor}{rgb}{1, 0, 1}
\definecolor{errorcolor}{rgb}{1, 0, 0}
\newenvironment{knitrout}{}{} % an empty environment to be redefined in TeX

\usepackage{alltt}  % [t], [c], или [b] --- вертикальное выравнивание на слайдах (верх, центр, низ)
%\documentclass[handout]{beamer} % Раздаточный материал (на слайдах всё сразу)
%\documentclass[aspectratio=169]{beamer} % Соотношение сторон

%\usetheme{Berkeley} % Тема оформления
%\usetheme{Bergen}
%\usetheme{Szeged}

%\usecolortheme{beaver} % Цветовая схема
%\useinnertheme{circles}
%\useinnertheme{rectangles}

%\usetheme{Madrid}
\usetheme{HSE}

%%% Работа с русским языком
\usepackage[cm-default]{fontspec}
\usepackage{xunicode}
\usepackage{xltxtra}
%\usepackage{polyglossia}
%\setdefaultlanguage[spelling=modern]{russian}
%\setotherlanguage{english}

\setmainfont{Times New Roman}
\setsansfont{Arial}
\setmonofont{Inconsolata}

%\usepackage{pscyr}
\usepackage{cmap}					% поиск в PDF
\usepackage{mathtext} 				% русские буквы в формулах
\usepackage[T2A]{fontenc}			% кодировка
\usepackage[utf8]{inputenc}			% кодировка исходного текста
\usepackage[english,russian]{babel}	% локализация и переносы

%% Beamer по-русски
\newtheorem{rtheorem}{Теорема}
\newtheorem{rproof}{Доказательство}
\newtheorem{rexample}{Пример}

%%% Дополнительная работа с математикой
\usepackage{amsmath,amsfonts,amssymb,amsthm,mathtools} % AMS
\usepackage{icomma} % "Умная" запятая: $0,2$ --- число, $0, 2$ --- перечисление

%% Номера формул
%\mathtoolsset{showonlyrefs=true} % Показывать номера только у тех формул, на которые есть \eqref{} в тексте.
%\usepackage{leqno} % Нумерация формул слева

%% Свои команды
\DeclareMathOperator{\sgn}{\mathop{sgn}}

%% Перенос знаков в формулах (по Львовскому)
\newcommand*{\hm}[1]{#1\nobreak\discretionary{}
	{\hbox{$\mathsurround=0pt #1$}}{}}

%%% Работа с картинками
\usepackage{graphicx}  % Для вставки рисунков
\graphicspath{{images/}{images2/}}  % папки с картинками
\setlength\fboxsep{3pt} % Отступ рамки \fbox{} от рисунка
\setlength\fboxrule{1pt} % Толщина линий рамки \fbox{}
\usepackage{wrapfig} % Обтекание рисунков текстом
\usepackage{caption}

\usepackage{pgffor}% http://ctan.org/pkg/pgffor

%%% Работа с таблицами
\usepackage{array,tabularx,tabulary,booktabs} % Дополнительная работа с таблицами
\usepackage{longtable}  % Длинные таблицы
\usepackage{multirow} % Слияние строк в таблице

%%% Программирование
\usepackage{etoolbox} % логические операторы

%%% Другие пакеты
\usepackage{lastpage} % Узнать, сколько всего страниц в документе.
\usepackage{soul} % Модификаторы начертания
\usepackage{csquotes} % Еще инструменты для ссылок
%\usepackage[style=authoryear,maxcitenames=2,backend=biber,sorting=nty]{biblatex}
\usepackage{multicol} % Несколько колонок

%%% Картинки
\usepackage{tikz} % Работа с графикой
\usepackage{pgfplots}
\usepackage{pgfplotstable}

\usepackage{ragged2e}


\usetikzlibrary{quotes,arrows.meta}
\tikzset{
	annotated cuboid/.pic={
		\tikzset{%
			every edge quotes/.append style={midway, auto},
			/cuboid/.cd,
			#1
		}
		\draw [every edge/.append style={pic actions, densely dashed, opacity=.5}, pic actions]
		(0,0,0) coordinate (o) -- ++(-\cubescale*\cubex,0,0) coordinate (a) -- ++(0,-\cubescale*\cubey,0) coordinate (b) edge coordinate [pos=1] (g) ++(0,0,-\cubescale*\cubez)  -- ++(\cubescale*\cubex,0,0) coordinate (c) -- cycle
		(o) -- ++(0,0,-\cubescale*\cubez) coordinate (d) -- ++(0,-\cubescale*\cubey,0) coordinate (e) edge (g) -- (c) -- cycle
		(o) -- (a) -- ++(0,0,-\cubescale*\cubez) coordinate (f) edge (g) -- (d) -- cycle;
		\path [every edge/.append style={pic actions, |-|}]
		(b) +(0,-5pt) coordinate (b1) edge ["\cubex \cubeunits"'] (b1 -| c)
		(b) +(-5pt,0) coordinate (b2) edge ["\cubey \cubeunits"] (b2 |- a)
		(c) +(3.5pt,-3.5pt) coordinate (c2) edge ["\cubez \cubeunits"'] ([xshift=3.5pt,yshift=-3.5pt]e)
		;
	},
	/cuboid/.search also={/tikz},
	/cuboid/.cd,
	width/.store in=\cubex,
	height/.store in=\cubey,
	depth/.store in=\cubez,
	units/.store in=\cubeunits,
	scale/.store in=\cubescale,
	width=10,
	height=10,
	depth=10,
	units=cm,
	scale=.1,
}



\usepackage{enumitem}
\newcommand*\circled[1]{\tikz[baseline=(char.base)]{\node[shape=circle,draw,inner sep=2pt] (char) {#1};}}
%\begin{enumerate}[label=\protect\circled{\arabic*}]
% кружки можно отменить в отдельном списке с помошью переобозначения:
%	\def\labelenumi{\arabic{enumi}.}


\title{Анализ возрастаний потоков заряженных~частиц в авроральных областях по результатам эксперимента ДЭПРОН}
\subtitle{Семинар НИИЯФ ОКН}
\author[Иван~Золотарев] {И.А. Золотарев, В.В. Бенгин, О.Ю. Нечаев,М.И. Панасюк, В.Л. Петров, И.В. Яшин,  А.М. Амелюшкин } 

\date{\today}
\institute[SINP MSU]{Skobeltsyn Institute of Nuclear Physics \\M.V. Lomonosov Moscow State University}












\begin{document}
	
\frame[plain]{\titlepage}	% Титульный слайд


%\begin{frame}{Overview}
%	\tableofcontents
%\end{frame}

%\tableofcontents[h\textsc{}ideallsubsections]
%%%%%%%%%%%%%%%%%%%%%%%%%%%%%%%%%%%%%%%%%%%%%
%\setbeamercolor{background canvas}{bg=black}
\section{ЛОМОНОСОВ}
\subsection{российский университетский спутник}
\begin{frame}	

\frametitle{\insertsection} 
\framesubtitle{\insertsubsection} 

\centering
\includegraphics[width=0.85\linewidth]{figures/lomo3}

 
\includegraphics[width=0.85\linewidth]{images/pic/pics}

%\includegraphics[width=0.85\linewidth]{images/pic/screenshot001}
%                                          
%\includegraphics[width=0.05\linewidth]{images/pic/math}\hspace{2mm}
%\includegraphics[width=0.05\linewidth]{images/pic/msu}
%\includegraphics[width=0.05\linewidth]{images/pic/niiem}
%\includegraphics[width=0.05\linewidth]{images/pic/roskosmos}
%\includegraphics[width=0.05\linewidth]{images/pic/sai}
%\includegraphics[width=0.05\linewidth]{images/pic/sinp}
%\includegraphics[width=0.05\linewidth]{images/pic/sungkyunkwan}
%\includegraphics[width=0.05\linewidth]{images/pic/symbol-lg}
%\includegraphics[width=0.05\linewidth]{images/pic/symbol-sm}
%\includegraphics[width=0.05\linewidth]{images/pic/ucla}
%\includegraphics[width=0.05\linewidth]{images/pic/vniiem}
%\includegraphics[width=0.05\linewidth]{images/pic/yonsei}
%\includegraphics[width=0.05\linewidth]{images/pic/buap}
%\includegraphics[width=0.05\linewidth]{images/pic/imec}
%\includegraphics[width=0.05\linewidth]{images/pic/jinr}
\end{frame}
%\setbeamercolor{background canvas}{bg=white}
%%%%%%%%%%%%%%%%%%%%%%%%%%%%%%%%%%%%%%%%%%%%%

\section{Задача прибора ДЭПРОН}
\subsection{Радиационный мониторинг на высокоширотных орбитах}
\begin{frame}	

\frametitle{\insertsection} 
\framesubtitle{\insertsubsection} 
	\begin{center}
		\includegraphics[width=0.9\linewidth]{figures/image7}
	\end{center}

{\tiny Ранее доложенные результаты:	
\begin{itemize}
	\item Географические распределения потоков частиц  и мощности дозы
	 \item Распределения в геомагнитных координатах
\end{itemize}}
\end{frame}
%%%%%%%%%%%%%%%%%%%%%%%%%%%%%%%%%%%%%%%%%%%%%

\section{Всплески интенсивности}
\begin{frame}	

\frametitle{\insertsection} 

\centering
\includegraphics[width=0.55\linewidth]{images/depronseclognew}

Временные серии данных показывают всплеск   и прохождение отрогов пояса или
аномалии
\end{frame}
%%%%%%%%%%%%%%%%%%%%%%%%%%%%%%%%%%%%%%%%%%%%%
\section{История исследования}

\begin{frame}	
\frametitle{\insertsection} 
{\tiny 
Характерные публикации по теме возрастаний потоков частиц в
высокоширотных областях:

\begin{enumerate}[label=\protect\circled{\arabic*}]
	\item
	Вернов С.Н., Чудаков А.Е. \\ Исследования космических лучей и земного корпускулярного излучения при полетах ракет и спутников // УФН. 1960. № 4 (70). C. 585.
	\item
	Базилевская Г.А., Калинин М.С., Квашнин А.Н., Крайнев М.Б., Махмутов В.С., Свиржевская А.К., Свиржевский Н.С., Стожков Ю.И., Балабин Ю.В., Гвоздевский Б.Б. \\ Высыпания высокоэнергичных магнитосферных электронов и сопутствующие характеристики солнечного ветра // Геомагнетизм и Аэрономия. 2017. № 2 (57). C. 164–172.	
	\item
	\textbf{Dachev T. P., 	Tomov B. ,	Matviichuk Yu,	Dimitrov Pl, Bankov N.\\ Relativistic electrons high doses at International Space Station and Foton M2/M3 satellites // Advances in Space Research. 2009. № 12 (44). C. 1433–1440.}
	\item
	Hendry, A. T., C. J. Rodger, M. A. Clilverd, M. J. Engebretson, I. R. Mann, M. R. Lessard, T. Raita, and D. K. Milling \\ Confirmation of EMIC wave-driven relativistic electron precipitation // Journal of Geophysical Research: Space Physics. 2016. № 6 (121). C. 5366–5383.
	\item Horne R. B.,  Glauert S. A., 	Meredith N. P., 	Boscher D., 	Maget V., Heynderickx D. ,	Pitchford D. \\ Space weather impacts on satellites and forecasting the Earth’s electron radiation belts with SPACECAST // Space Weather. 2013. № 4 (11). C. 169–186.
\end{enumerate}
}
{\tiny 


Новизна нашего исследования заключается в оценке дозиметрических характеристик всплесков.

Если кто то из коллег осведомлен о публикациях дозиметрических характеристик описанных всплесков, мы будем очень благодарны за указание таких работ.
}



\end{frame}

%%%%%%%%%%%%%%%%%%%%%%%%%%%%%%%%%%%%%%%%%%%%%
\section{Истории изучения вопроса}

\begin{frame}	
\frametitle{\insertsection} 

{\tiny Даже на небольших высотах, начиная от 300 км в интервале геомагнитных широт 55-70, наблюдается резкое возрастание интенсивности излучения и частицами, составляющими этот внешний радиационный пояс, являются электроны различных энергий, и поток  с энергиями от 1 до 1~МэВ достигает 10\textsuperscript{5}см\textsuperscript{-2}сек\textsuperscript{-1}стер\textsuperscript{-1} [Вернов С.Н. 1960].  Вернов выдвигает предположение о двух энергетически разобщенных группах электронов --- с энергией в десятки кэВ и единицы МэВ. 
	
Базилевская Г.А. в свои работах отмечает связь длительных периодов отрицательных значений $ B_z $ с высыпаниями высокоэнергичных электронов.

 \includegraphics[width=0.6\linewidth]{images/poesemic}
	\includegraphics[width=0.4\linewidth]{mltrep}

 
\begin{itemize}
	\item Высыпания электронов по данным POES, скорее всего вызваны волнами EMIC, обнаруживаемыми на поверхности;
	\item EMIC-IPDP неоднократно срабатывал при  MLT,  совпадающим с обнаруженными на POES  высыпаниями электронов;
	\item Волны EMIC, обнаруженные при  высыпаниях электронов, представляют собой в основном волны полосы IPDP.
\end{itemize}
}
\end{frame}
	
%%%%%%%%%%%%%%%%%%%%%%%%%%%%%%%%%%%%%%%%%%%%%
\begin{frame}
\frametitle{\insertsection} 

\begin{columns}[T]
\begin{column}{.5\textwidth}
{\tiny 
По сообщению Dachev T. P. [2009], измеренные абсолютные максимумы мощности дозы	по релятивистским электронам:
\begin{itemize}
	\item  304 мкГр/ч за защитой  1,75 г см-2,  Foton M2
	 \item  2314 мкГр/ч за защитой   0,71 г см2, экранируя в Foton M3 
	 \item  19 195 мкГр/ч за защитой менее 0,4 г / см2 на МКС (поток составляет 8363 см-2*с-1) .
\end{itemize}
}
\end{column}
\begin{column}{.5\textwidth}
	\begin{center}
		\includegraphics[width=1\linewidth]{images/dachevrep}
	\end{center}
\end{column}
\end{columns}
\end{frame}

%%%%%%%%%%%%%%%%%%%%%%%%%%%%%%%%%%%%%%%%%%%%%
\section{План доклада}\label{header-n26}
\begin{frame}	
\frametitle{\insertsection} 
\begin{enumerate}[label=\protect\circled{\arabic*}]

	\item
	Описание прибора ДЭПРОН
	\item
	Алгоритм обработки данных
	\item
	Доступность данных и порядок наземной обработки
	\item
	Результаты без всплесков
	\item
	Статистика всплесков и их феноменология. Критерии отбора событий.
	\item
	Географическое распределение всплесков
	\item
	Связь с параметрами солнечной активности
	\item
	Дозиметрические характеристики всплесков
\end{enumerate}

\end{frame}

%%%%%%%%%%%%%%%%%%%%%%%%%%%%%%%%%%%%%%%%%%%%%
%\begin{frame}
%\frametitle{Содержание}
%\tableofcontents
%\end{frame}


\section{ДЭПРОН}
\subsection{Детекторная система}

%%%%%%%%%%%%%%%%%%%%%%%%%%%%%%%%%%%%%%%%%%%%%
\begin{frame}
\frametitle{\insertsection} 
\framesubtitle{Коэффициенты перехода от внутренних единиц к потоку и
дозе. Схема расположения детекторов прибора и защиты вокруг них, минимальные энергии проникающих частиц.}
\begin{columns}[T]
	\begin{column}{.5\textwidth}
		\begin{block}{	 }		
		\tiny ДЭПРОН - Дозиметр Электронов, ПРОтонов и Нейтральных частиц 	
	 		\begin{enumerate}[label=\protect\circled{\arabic*}]
				\item Корпус --- 1,9 мм алюминия, Д16т;
				\item  Бериллиевая бронза --- фольга 50 мкм;
				\item[] Детекторы:				
				\begin{enumerate}
					\tiny
					\item[D1] Детектор 	--- 0,3 мм кремний
					\item[Мl] Модератор  	--- 0,3 мм алюминий
					\item[D2] Детектор 	--- 0,3 мм кремний
					\item[D3] \textbf{He-3} счетчик	
					\item[D4] \textbf{He-3} с защитой 1~см оргстекла
				\end{enumerate}
			\end{enumerate}
		
		\tiny Объемная модель
		
		\includegraphics[width=0.7\textwidth]{images/deproncatia2}
		
		\end{block}
	\end{column}
	\begin{column}{.5\textwidth}
		\begin{block}{}
			% Your image included here
			\tiny Схема расположения детекторов и защиты
		
%			\vspace{0.5cm}
			\includegraphics[width=0.2\textwidth]{images/cartesiancoordinatesystem.png}
			
			\begin{tikzpicture}[scale=0.55, transform shape]
			\pic [fill=gray, text=blue, draw=gray] at (6,-3.0) {annotated cuboid={width=60, height=1.5, depth=60, units=мм}};
			
			\pic [fill=orange, text=blue, draw=blue] at (6,0.6) {annotated cuboid={width=50, height=0.05, depth=30, units=мм}};
			
			\pic [fill=orange, text=blue, draw=blue] at (6,0.6) {annotated cuboid={width=50, height=0.05, depth=30, units=мм}};
			
			\pic [fill=orange, text=blue, draw=blue] at (6,-1.4) {annotated cuboid={width=50, height=0.05, depth=30, units=мм}};
				
			\pic [fill=gray, text=blue, draw=gray] at (6,1.4) {annotated cuboid={width=60, height=1.9, depth=60, units=мм}};
			

			\pic [fill=gray, text=blue, draw=gray] at (6,-1.0) {annotated cuboid={width=50, height=1, depth=30, units=мм}};

			\pic [fill=brown, text=blue, draw=blue] at (5,0) {annotated cuboid={width=10, height=0.3, depth=10, units=мм}};
			
			
			\pic [fill=brown, text=green!50!black, draw=green!25!black] at (5,-2.0) {annotated cuboid={width=10, height=0.3, depth=10, units=мм}};
			
			\draw[|->,semithick] (7,-1.8) -- (7,0);
			\draw (8,-1.8) node {$z=0$};
			\draw (8,0) node {$z=20$};
			\end{tikzpicture}
		{\tiny 
			Массовая защита по осям прибора, г·см-2 :
		\begin{tabular}{l|c|c|c|c|c}
			+ Х & -Х  &  +Y  & -Y  & +Z  & -Z   \\ \hline
			1,1 & 4,0 & >8,0 & 2,7 & 1,6 & 4,3 \\
		\end{tabular} 
	}
			
		\end{block}
	\end{column}
\end{columns}
\end{frame}

%%%%%%%%%%%%%%%%%%%%%%%%%%%%%%%%%%%%%%%%%%%%%
\begin{frame}
\frametitle{\insertsection} 
\framesubtitle{\insertsubsection}
\begin{columns}[T]
\begin{column}{.5\textwidth}
	\begin{block}{}			
		%DEPRON - Dosimeter of Electrons, PROtons and Neutral particles
		%\includegraphics[width=0.7\linewidth]{images/04062010072}
		{\tiny
		Наиболее чувствительный информационный параметр при работе ДЭПРОН --- скорость счета верхнего детектора. Проведена оценка минимальной энергии заряженных частиц, к которым данный детектор чувствителен. Так как детектор закрыт сверху алюминиевой крышкой толщиной около 2~мм, он должен быть чувствителен к протонам с энергией больше 18~МэВ и электронам с энергией больше примерно 1~МэВ, а также - возможно - к тормозному излучению. Порог дискриминации сигналов с детектора около 100~КэВ. 
		
		Второй детектор c дополнительной пластиной 1~мм Алюминия должен иметь границу чувствительности 1,5~МэВ по электронам и 23~МэВ по протонам.
		%Тем не менее вопрос уточнения границы чувствительности по минимальным энергиям продолжает оставаться важным и на первом этапе были проведены оценки с помощью данных по проникновению электронов и протонов с сайта NIST \cite{NIST}, физическая модель лежащая в основе этих данных основывается на теории Бете\cite{Bethe1930} с поправкой Штернхаймера \cite{Sternheimer1952} на плотность вещества и подробно описана в ряде статей этого института \cite{Bichsel1992, Ashley1972}.
		
		%Пользуясь представленными зависимостями для уточненной минимальной толщины корпуса прибора, которая составляет 2,5~мм, что соответствует 0,65~г/см\textsuperscript{2}, была повышена предварительная оценка порога нижних энергий, которые способен регистрировать ДЭПРОН по электронам до 1~МэВ и по протонам до 20~МэВ. Для ядер гелия прибор чувствителен начиная с 90~МэВ. Так как эти зависимости могут использоваться только для средних пробегов частиц, для оценки функции энергетической чувствительности требуется более подробный анализ, который может быть произведен с помощью Монте-Карло моделирования.
		
	}
	\end{block}
\end{column}
\begin{column}{.5\textwidth}
	\begin{block}{}
	\begin{center}
		\includegraphics[width=0.5\linewidth]{images/edata}
		\includegraphics[width=0.5\linewidth]{images/pdata}
	%	\includegraphics[width=0.32\linewidth]{images/adata}
	\end{center}

	\tiny{Графики средних пробегов заряженных частиц для алюминия. Представлены величины: \\		``CSDA range'' --- глубина в приближении непрерывного замедления \\		``Projected range'' --- среднее значение глубины, на которую заряженная частица проникает в процессе замедления до остановки}

	\end{block}
\end{column}
\end{columns}
\end{frame}
%%%%%%%%%%%%%%%%%%%%%%%%%%%%%%%%%%%%%%%%%%%%%


\begin{frame}	
\frametitle{\insertsection} 
\framesubtitle{Чувствительность нейтронных счетчиков, мне кажется не нужна в этой презентации}

	\includegraphics[width=0.5\linewidth]{images/nsens3}
	\includegraphics[width=0.5\linewidth]{images/nsens4}
%	\caption{Профили чувствительности нейтронных счетчиков по итогам моделирования прибора Дэпрон. }


\tiny{Профили чувствительности нейтронных счетчиков, отражают отношение зарегистрированных в счетчиках нейтронов к потоку нейтронов, прошедших через объем счетчика. Эта величина соответствует функции чувствительности. Фактом регистрации нейтрона в детекторе при моделировании считалось энерговыделение в объеме заполняющего газа более 500~кэВ. При сравнении профилей чувствительности не защищенного и окруженного оргстеклом нейтронных детекторов можно заметить что пик чувствительности более защищенного детектора находится на 0,005~эВ, а для защищенного этот пик находится на 0,5~эВ энергии нейтронов.}

\end{frame}
%%%%%%%%%%%%%%%%%%%%%%%%%%%%%%%%%%%%%%%%%%%%%
\section{Алгоритм обработки данных}

\begin{frame}	
	\frametitle{\insertsection} 
	\framesubtitle{Особенности алгоритма обработки данных для поиска всплесков}
	\centering
	
	\includegraphics[width=0.7\linewidth]{images/Flash/Rplot03.png}
	
{\tiny 	Временное распределение скорости счета в зарегистрированных всплесков. Общее число выделенных всплесков за время работы прибора ДЭПРОН достигает 90.
	
	
	 \begin{flushleft}
	 	Были найдены и выделены возрастания скоростей счета в первом детекторе, превышающие по абсолютной величине 5000~отсчетов в секунду, что соответствует более 700~с\textsuperscript{-1} см\textsuperscript{-1} стер\textsuperscript{-1}. 	Такая величина скорости счета была выбрана исходя из быстродействия прибора ДЭПРОН.
 \end{flushleft}
}
\end{frame}

%%%%%%%%%%%%%%%%%%%%%%%%%%%%%%%%%%%%%%%%%%%%%
\section{Доступность данных}

\begin{frame}	
\frametitle{\insertsection} 
\framesubtitle{Доступность данных и порядок наземной обработки}
{\tiny В результате полугода работы прибора ДЭПРОН накоплено информации - 35 тысяч файлов бинарных данных, общим объемом 141 Мбайт.

 Эти данные являются сжатыми и после распаковки в человеко-читаемый табличный вид представлены массивом файлов объемом 1,3 Гб, они собраны в файлы по дням года и осуществлена привязка данных к географическим и геомагнитным координатам}.
\begin{center}
	\includegraphics[width=1\linewidth]{images/timegraph}
\end{center}
\end{frame}
%%%%%%%%%%%%%%%%%%%%%%%%%%%%%%%%%%%%%%%%%%%%%

\section{Разделение участков траектории}\label{header-n0}
\begin{frame}	

\frametitle{\insertsection} 

\centering
\includegraphics[width=0.55\linewidth]{images/depronseclognew41}

{\tiny 
Временные серии с выделением прохождений аномалии и отрогов внешнего пояса.

 Далее разделение проводилось автоматически по геомагнитным координатам:\\  аномалия ($ b < 0.21 $) и отроги внешнего пояса ($ 3 < L < 7 $), остальное --- регулярный полет по орбите.}
\end{frame}
%%%%%%%%%%%%%%%%%%%%%%%%%%%%%%%%%%%%%%%%%%%%%
\section{Результаты измерений дозиметра}
\subsection{Суточные дозы}

\begin{frame}	
\frametitle{\insertsection} 
\framesubtitle{\insertsubsection} 

\begin{center}
	\includegraphics[width=0.5\linewidth]{ressumdaydose1}
	\includegraphics[width=0.5\linewidth]{ressumdaydose2}
\end{center}

{\tiny Интегральные суточные дозы в аномалии и в полярных областях. Каждая точка представляет один день. 
	
}
\end{frame}
%%%%%%%%%%%%%%%%%%%%%%%%%%%%%%%%%%%%%%%%%%%%%
\subsection{Максимальные мощности дозы}
\begin{frame}	
\frametitle{\insertsection} 
\framesubtitle{\insertsubsection} 

\begin{center}
		
	\includegraphics[width=0.45\linewidth]{resmaxdosepolaranom1}
	\includegraphics[width=0.45\linewidth]{resmaxdosepolaranom2}
\end{center}


{\tiny Максимальные мощности дозы в аномалии и в полярных областях. Каждая точка представляет один день. В этих результатах не проведено отделение всплесков, поэтому значительный разброс в полярных областях связан с накладывающимися всплесками.  Именно этот разброс и привлекает наше внимание.
\[ 
\begin{array}{lcc}
 & $ Верхний детектор, мГр/ч $ &$  Нижний детектор, мГр/ч $\\
$ Аномалия $	& 5,1 & 3,8 \\ 
$ Авроральная зона $	& 26 & 3 \\
$ ГКЛ $	& 7,2 & 2,2
\end{array} \] 
 }
\end{frame}


%%%%%%%%%%%%%%%%%%%%%%%%%%%%%%%%%%%%%%%%%%%%%
%\begin{frame}	
%\frametitle{\insertsection} 
%\framesubtitle{\insertsubsection} 
%
%\begin{center}
%	\includegraphics[width=0.6\linewidth]{images/resmaxdosepolaranomboth}
%\end{center}
%
%{\tiny Максимальные мощности дозы в аномалии и в полярных областях. Каждая точка представляет один день. Яркие точки ---- полярные области, бледные---  аномалия. 
%
%}
%\end{frame}

%%%%%%%%%%%%%%%%%%%%%%%%%%%%%%%%%%%%%%%%%%%%%
\section{Статистика всплесков и их феноменология. }
%%%%%%%%%%%%%%%%%%%%%%%%%%%%%%%%%%%%%%%%%%%%%
\begin{frame}	
\frametitle{\insertsection} 
\framesubtitle{Особенности алгоритма обработки данных для поиска всплесков}
{\tiny Отбор всплесков интенсивности производился по превышению 5000 отсчетов в верхнем детекторе, что составляет  700 $ c^{-1}cm^{-2}sr^{-1} $ (геометрический фактор около 2$ \pi $).  Такие величины скоростей счета характерны только для авроральных областей:
\[ \begin{array}{l|rr}
& $ Верхний детектор, $ &$  Нижний детектор,  $ \\
& c^{-1}cm^{-2}sr^{-1} &c^{-1}cm^{-2}sr^{-1} \\
\hline
$ Аномалия $	& 112 & 66 \\ 
$ Авроральная зона $	& 2444 & 244 \\ 
$ ГКЛ $	& 733 & 45
\end{array} \] }
\begin{center}
	\includegraphics[width=0.45\linewidth]{images/resmaxcountpolaranomboth}
\end{center}

\end{frame}

%%%%%%%%%%%%%%%%%%%%%%%%%%%%%%%%%%%%%%%%%%%%%
\begin{frame}	
\frametitle{\insertsection} 
\framesubtitle{Соотношение программного и аппаратного счетчиков}
{\tiny Величины повышенных потоков, зарегистрированных в первом полупроводниковом детекторе, в среднем в 30-100 раз выше, чем во втором детекторе и при одновременной регистрации в двух детекторах.}

\centering


\begin{center}
	\includegraphics[width=0.55\linewidth]{images/Rplot04}
	\includegraphics[width=0.45\linewidth]{images/Rplot11}
\end{center}
\tiny{Слева: Соотношение программного и аппаратного счетчиков для моментов времени, в которые регистрировались всплески. Прямая на рисунке показывает соотношение, при котором все зарегистрированные частицы обработаны программой прибора. Каждая точка показывает интегральный счет за одну секунду.}

{\tiny Справа: Отношение счёта нижнего детектора к счету в верхнем детекторе, в зависимости от значений аппаратного~счетчика. Всплески можно разделить на две группы по проникающей способности частиц.
}


\end{frame}
%%%%%%%%%%%%%%%%%%%%%%%%%%%%%%%%%%%%%%%%%%%%%
\begin{frame}	
\frametitle{\insertsection} 
\framesubtitle{Критерии отбора событий.}

\begin{center}
	\includegraphics[width=0.7\linewidth]{images/Rplot12}
	
{\tiny 
	Распределение наблюдаемых всплесков по длительности.
	\\ Две трети всплесков имеют длительность менее 20 секунд, что дает предполагать, что характерные размеры областей повышенного счета --- \~150~км}
\end{center}


\end{frame}
%%%%%%%%%%%%%%%%%%%%%%%%%%%%%%%%%%%%%%%%%%%%%
\begin{frame}	
\frametitle{\insertsection} 
\framesubtitle{Характерный всплеск 1}


	\centering
	\includegraphics[width=0.49\linewidth]{images/flash021116}
	\includegraphics[width=0.49\linewidth]{images/flash021116big}
	
	\tiny{Временной профиль возрастания, зарегистрированный 2 Ноября 2016 года в 00:40 по мировому времени. Он отличается необычно высокой дозой во втором детекторе.
	}

\end{frame}
%%%%%%%%%%%%%%%%%%%%%%%%%%%%%%%%%%%%%%%%%%%%%
\begin{frame}	
\frametitle{\insertsection} 
\framesubtitle{Характерный всплеск 2}

\tiny{

	Для всплеска, зарегистрированного в 2016-10-28 21:43:42 --- 21:45:26~UTC поглощенная доза для верхнего детектора превышает 0,35~мГр, это максимальное значение для всех рассмотренных возрастаний. Данный всплеск отличается наибольшей по сравнению с другими всплесками продолжительностью - 105 секунд.}

\begin{columns}
	\begin{column}[t]{.5\textwidth}
	\centering 	
	\includegraphics[width=1\linewidth]{images/flash281016}
	
	 Всплеск
	\end{column}
	\hfill
	\begin{column}[t]{.5\textwidth}
	\includegraphics[width=1\linewidth]{images/anomalycross}
	
	\centering Аномалия
	\end{column}%
\end{columns}
\begin{center}
		

\end{center}


\end{frame}
%%%%%%%%%%%%%%%%%%%%%%%%%%%%%%%%%%%%%%%%%%%%%
\section{Географическое распределение всплесков}

\begin{frame}	
\frametitle{\insertsection} 
\begin{center}
	\includegraphics[width=1.1\linewidth]{images/flashmap}
\end{center}


\end{frame}

%%%%%%%%%%%%%%%%%%%%%%%%%%%%%%%%%%%%%%%%%%%%%
\section{Распределение всплесков в геомагнитных~координатах}

\begin{frame}	
\frametitle{\insertsection} 
{\tiny Всплески наблюдаются в основном с 21 часа и по мере приближения к полночи их число уменьшается. 	Распределение сильно отличается от MLT наблюдаемых при прохождениях авроральных областей.}
	\includegraphics[width=0.38\linewidth,trim={0 0 3.2cm 0}, clip]{images/flashmltn}
	\includegraphics[width=0.51\linewidth]{images/flashmlts}\\	
	\tiny{Магнитное локальное время для зарегистрированных за все время всплесков.}

\end{frame}
%%%%%%%%%%%%%%%%%%%%%%%%%%%%%%%%%%%%%%%%%%%%%
\section{Распределение пролетов авроральной области в геомагнитных~координатах}

\begin{frame}	
\frametitle{\insertsection} 

\includegraphics[width=0.3\linewidth]{images/daymltn}
\includegraphics[width=0.3\linewidth]{images/daymlts}
\includegraphics[width=0.38\linewidth]{images/Rplot020}
%\includegraphics[width=0.3\linewidth]{images/Rplot020}	\\

{ Наблюдения различных значений магнитного локального времени в точках с координатами пролета спутника Ломоносов за период времени в одни сутки для полярных областей. 
}

\end{frame}
%%%%%%%%%%%%%%%%%%%%%%%%%%%%%%%%%%%%%%%%%%%%%
\section{Связь всплесков с параметрами авроральной активности}

\begin{frame}	
\frametitle{\insertsection} 
%\centering
%
%\includegraphics[width=0.7\linewidth, trim={5 2cm 0 0}, clip]{images/Flash/Rplot03.png}
%
%\includegraphics[width=0.7\linewidth, trim={0 1.5cm 0 1cm}, clip]{images/Flash/Rplot01}
\begin{center}
	\includegraphics[width=0.55\linewidth]{images/flashAEbz}
\end{center}
\tiny{Временное распределение зарегистрированных всплесков. Общее число выделенных всплесков за время работы прибора ДЭПРОН достигает 90.\\ Второй график показывает вариации индекса AE, при отрицательном Bz. Данные предоставлены сервисом  \href{https://omniweb.gsfc.nasa.gov/}{OMNIweb}.}

\end{frame}
%%%%%%%%%%%%%%%%%%%%%%%%%%%%%%%%%%%%%%%%%%%%%
\section{Дозиметрические характеристики всплесков}

\begin{frame}	
\frametitle{\insertsection} 
\begin{center}
	\includegraphics[width=1\linewidth]{images/dose2vsdose1}
\end{center}


\end{frame}
%%%%%%%%%%%%%%%%%%%%%%%%%%%%%%%%%%%%%%%%%%%%%
\section{Дозиметрические характеристики всплесков}

\begin{frame}	
\frametitle{\insertsection} 
\begin{center}
	\includegraphics[width=0.9\linewidth]{images/dosenorm}
\end{center}


\end{frame}
%%%%%%%%%%%%%%%%%%%%%%%%%%%%%%%%%%%%%%%%%%%%%

\begin{frame}	
\frametitle{\insertsection} 
\begin{center}
	
	\includegraphics[width=.9\linewidth]{dosevscounth}
\end{center}

\end{frame}
%%%%%%%%%%%%%%%%%%%%%%%%%%%%%%%%%%%%%%%%%%%%%

\begin{frame}	
\frametitle{\insertsection} 
Максимальные мощности дозы во всплесках, по сравнению с максимальными мощностями в полярных областях. 


\begin{center}
	\includegraphics[width=0.5\linewidth]{images/resmaxpolarflash}
	\includegraphics[width=0.5\linewidth]{images/resmaxpolarflash2}
\end{center}
{\tiny 
Синие точки представляют суточные максимумы для всплесков, а красные максимумы за весь день. На левом графике синие точки почти всегда накладываются на красные.

Поэтому  можно обозначить <<Вершинами айсберга>>, то как соотносятся максимальные мощности дозы во всплесках и за день. Для второго детектора такой характер наблюдается только для части случаев. }

\end{frame}
%%%%%%%%%%%%%%%%%%%%%%%%%%%%%%%%%%%%%%%%%%%%%

\begin{frame}	
\frametitle{\insertsection} 
{\tiny Суммарные мощности дозы за всплеск, по сравнению с интегральными суточными мощностями в полярных областях. }


\begin{center}
	\includegraphics[width=0.5\linewidth]{images/ressumdosepolarflash1sci}
	\includegraphics[width=0.5\linewidth]{images/ressumdosepolarflash2sci}
\end{center}


\end{frame}
%%%%%%%%%%%%%%%%%%%%%%%%%%%%%%%%%%%%%%%%%%%%%

\begin{frame}	
\frametitle{\insertsection} 
{\tiny Суммарные мощности дозы за всплеск, по сравнению с интегральными суточными мощностями в полярных областях. }


\begin{center}
	\includegraphics[width=0.5\linewidth]{images/ressumdosepolarflash1lin}
	\includegraphics[width=0.5\linewidth]{images/ressumdosepolarflash2lin}
\end{center}


\end{frame}
%%%%%%%%%%%%%%%%%%%%%%%%%%%%%%%%%%%%%%%%%%%%%

\begin{frame}	
\frametitle{\insertsection} 
{\tiny Вклад   всплесков в суточную дозу в полярных областях }


\begin{center}
	\includegraphics[width=0.6\linewidth]{images/ressumdoseratio}
\end{center}


{\tiny Мы можем утверждать, что короткие всплески вносят существенный вклад в суммарную дозу в обоих детекторах. Этот вклад более значителен в полярных областях и достигает четверти величины для верхнего детектора}
\end{frame}
%%%%%%%%%%%%%%%%%%%%%%%%%%%%%%%%%%%%%%%%%%%%%
\section{Заключение}

\begin{frame}	
\frametitle{\insertsection} 
\begin{enumerate}[label=\protect\circled{\arabic*}]
	\item Всплески наблюдаются в основном с 21 часа и по мере приближения к полночи их число уменьшается; 	
	\item Наблюдаются ассиметрия "утро-вечер" в регистрации всплесков;
	\item Максимальные мощности дозы в время всплесков являются максимумами мощности дозы за день в полярной области для обоих детекторов;
	\item Суточные дозы в авроральных областях испытывают увеличение более чем в 5 раз, совпадающее по времени с регистрацией всплесков;
	\item Вклад в суточные дозы в авроральных областях от всплесков не превышает 25\%.
\end{enumerate}

\end{frame}

\end{document}
%%%%%%%%%%%%%%%%%%%%%%%%%%%%%%%%%%%%%%%%%%%%ы
%%%%%%%%%%%%%%%%%%%%%%%%%%%%%%%%%%%%%%%%%%%%%

\begin{frame}
\frametitle{\insertsection} 
\framesubtitle{\insertsubsection}
\begin{columns}[T]
	\begin{column}{.5\textwidth}
		\begin{block}{}
			
			\begin{figure}
				\centering
				\includegraphics[width=0.9\linewidth]{images/lomo3}
				\caption{Расположение прибора на космическом аппарате}
				\label{fig:lomo3}
			\end{figure}
			
		\end{block}
	\end{column}
	\begin{column}{.5\textwidth}
		\begin{block}{}
			% Your image included here
			
			\begin{figure}
				\centering
				\includegraphics[width=0.6\linewidth]{images/Depronoutside1}
				\caption{}
				\label{fig:depronoutside}
			\end{figure}
			
		\end{block}
	\end{column}
\end{columns}

\end{frame}


%------------------------------------------------		
\section{Dataset}
%------------------------------------------------	
\begin{frame}
	\frametitle{\insertsection} 
	\framesubtitle{\insertsubsection}
	{\small 	
	\justify{
	For the first half of year after the  Lomonosov satellite launch several switching on DEPRON was made according to the mission schedule.
	
	 Now the instrument is working in monitoring mode, with maximum power up time.
	
	During operation of the unit we received 25 thousand data sets containing 100 Mbytes of binary data.	}}

	
	\begin{columns}[T]
		\begin{column}{.5\textwidth}
			\begin{block}{Resulting datasets obtained by decompressing binary data files:}				
			\begin{itemize}
				
				\item  minutes timeseries
				\item  seconds timeseries
				\item  spectra files
				%\item  1 файл массивов высоких амплитуд
				%\item  1 файл массивов нейтронных вспышек
			\end{itemize}
			\end{block}
		\end{column}
		\begin{column}{.5\textwidth}
			\begin{block}{Decoding}
				% Your image included here
				\begin{figure}[th]
					\includegraphics[width=0.6\textwidth]{images/explorer}
				\end{figure}				
			\end{block}
		\end{column}
	\end{columns}
	

	
\end{frame}
	
	

\section{Mission First Results}

\subsection{Time scaling}	
%------------------------------------------------		
\begin{frame}
	\frametitle{\insertsection} 
	\framesubtitle{\insertsubsection}	
	\begin{center}
	
	\includegraphics[width=0.6\linewidth]{images/screenshot001}

	\includegraphics[width=0.6\linewidth]{images/screenshot002}

	\end{center}
\end{frame}

\subsection{Minute datasets}	
%------------------------------------------------		
\begin{frame}
	\frametitle{\insertsection} 
	\framesubtitle{\insertsubsection}	
	\begin{center}									
		\includegraphics[width=0.65\textwidth]{images/depron_min}%		\includegraphics[width=0.8\textwidth]{images/Rplot02}%

		
		{\footnotesize 
		}
	\end{center}
\end{frame}

\subsection{Seconds dataset}	
			
%------------------------------------------------		
\begin{frame}
	\frametitle{\insertsection} 
	\framesubtitle{\insertsubsection}
	\begin{center}									
		\includegraphics[width=1\textwidth]{images/Rplot05}%
		
		{\footnotesize 
		}
	\end{center}
\end{frame}




\subsection{Anomaly crossing}	
%------------------------------------------------		
\begin{frame}
	\frametitle{\insertsection} 
	\framesubtitle{\insertsubsection}
	\begin{center}		
%		Пример пролета аномалии 
%		кусок траектории с цветом интенсивности
%		Динамика потоков в первом и втором детекторе (без совп)
%		Динамика нейтронных скоростей счета по двум каналам
%		Временные диаграммы				
		
		\begin{columns}[t]
			\column{.5\textwidth}
			\centering
%			\includegraphics[width=5cm,height=4cm]{images/spe1anom}
			\includegraphics[width=5cm,height=4cm]{images/depron_sec_log08-25-16}
			\column{.5\textwidth}
			\centering
			\includegraphics[width=5cm,height=4cm]{images/depron_map_238}\\
			\includegraphics[width=5cm,height=3cm]{images/spe1anom}
		\end{columns}
	\end{center}
\end{frame}

\subsection{South auroral oval crossing}	
%------------------------------------------------		
\begin{frame}
	\frametitle{\insertsection} 
	\framesubtitle{\insertsubsection}
	\begin{center}
%		Пример пролета южной авроральной зоны
%		кусок траектории с цветом интенсивности
%		Динамика потоков в первом и втором детекторе (без совп)
%		Динамика нейтронных скоростей счета по двум каналам
%		Спектр энерговыделения за этот же промежуток времени.
%		Временные диаграммы									
		
		\begin{columns}[t]
			\column{.5\textwidth}
			\centering
%			\includegraphics[width=5cm,height=3.5cm]{images/spe1polar}\\
			\includegraphics[width=5cm,height=4cm]{images/depron_sec_log08-21-1611-35-00}
			\column{.5\textwidth}
			\centering
			\includegraphics[width=5cm,height=4cm]{images/depron_map_234}\\
			\includegraphics[width=5cm,height=3cm]{images/spe1polar}
		\end{columns}
	\end{center}
\end{frame}

\subsection{Timeseries}	
%------------------------------------------------		
\begin{frame}
	\frametitle{\insertsection} 
	\framesubtitle{\insertsubsection}
	\begin{center}		
								
		\includegraphics[width=0.8\textwidth]{images/sec_anom}%
		
		{\footnotesize 
		}
	\end{center}
\end{frame}


\subsection{22 23 августа 2016}
%------------------------------------------------		
\begin{frame}
	\frametitle{\insertsection} 
	\framesubtitle{\insertsubsection}
	Временные диаграммы за периоды 21 и 23 августа (данные левого рисунка заканчиваются в 7:30 22 августа)
	
	\begin{center}									
		\includegraphics[width=0.6\textwidth]{images/image4}%
		
		{\footnotesize 
		}
	\end{center}
\end{frame}

\subsection{20 августа 2016}	
%------------------------------------------------		
\begin{frame}
	\frametitle{\insertsection} 
	\framesubtitle{\insertsubsection}
\begin{center}		
	
	\includegraphics[width=0.8\textwidth]{images/image8}
	\includegraphics[width=0.2\linewidth]{images/Rplot}
\end{center}
\end{frame}
	

\subsection{21 августа 2016}	
%------------------------------------------------		
\begin{frame}
	\frametitle{\insertsection} 
	\framesubtitle{\insertsubsection}
\begin{center}		
	
	\includegraphics[width=0.8\textwidth]{images/image10}
	\includegraphics[width=0.2\linewidth]{images/Rplot}
\end{center}
\end{frame}
	

\subsection{23 августа 2016}	
%------------------------------------------------		
\begin{frame}
	\frametitle{\insertsection} 
	\framesubtitle{\insertsubsection}
\begin{center}		
	
	\includegraphics[width=0.8\textwidth]{images/image12}
	\includegraphics[width=0.2\linewidth]{images/Rplot}
\end{center}
\end{frame}

%------------------------------------------------		
\begin{frame}
	\frametitle{\insertsection} 
	\framesubtitle{\insertsubsection}
	\begin{center}		
		
		\includegraphics[width=0.8\textwidth]{images/image9}
		\includegraphics[width=0.2\linewidth]{images/Rplot}
	\end{center}
\end{frame}

%------------------------------------------------		
\begin{frame}
	\frametitle{\insertsection} 
	\framesubtitle{\insertsubsection}
	
	\begin{center}		
		
		\includegraphics[width=0.8\textwidth]{images/image11}
		\includegraphics[width=0.2\linewidth]{images/Rplot}
	\end{center}
\end{frame}	
%------------------------------------------------		
\begin{frame}
	\frametitle{\insertsection} 
	\framesubtitle{\insertsubsection}
\begin{center}		
	
	\includegraphics[width=0.8\textwidth]{images/image13}
	\includegraphics[width=0.2\textwidth]{images/Rplot}
\end{center}
\end{frame}


\subsection{Global Maps}	
%------------------------------------------------		
\begin{frame}
	\frametitle{\insertsection} 
	\framesubtitle{\insertsubsection}
	\begin{center}		
		\rotatebox{90}{Latitude}~
		\includegraphics[width=0.79\textwidth]{images/image7}
		\includegraphics[width=0.2\linewidth]{images/Rplot}\\
		Longitude
	\end{center}
\end{frame}
%------------------------------------------------		
\begin{frame}
	\frametitle{\insertsection} 
	\framesubtitle{\insertsubsection}
\begin{center}
	\rotatebox{90}{
			$ B $
	}\includegraphics[width=0.79\linewidth]{images/depron_lb}	
	\includegraphics[width=0.2\linewidth]{images/Rplot}\\
	$ L $
\end{center}
\end{frame}

%------------------------------------------------		
\begin{frame}
	\frametitle{\insertsection} 
	\framesubtitle{\insertsubsection}
\begin{center}
	\rotatebox{90}{
			$ B $
	}\includegraphics[width=0.79\linewidth]{images/depron_lb}	
	\includegraphics[width=0.2\linewidth]{images/Rplot}\\
	$ L $
\end{center}
\end{frame}
%------------------------------------------------		
\begin{frame}
	\frametitle{\insertsection} 
	\framesubtitle{\insertsubsection}
\begin{center}
\includegraphics[width=0.8\textheight]{images/depron_sec_log09-01-1600-00-00}
\end{center}
\end{frame}


%------------------------------------------------		
\begin{frame}
	\frametitle{\insertsection} 
	\framesubtitle{\insertsubsection}






	\begin{columns}[T]
		\begin{column}{.5\textwidth}
			\begin{block}{}
				\includegraphics[width=1\linewidth]{images/depron_lat_map_242}
				
				
				\includegraphics[width=1\linewidth]{images/depron_map_242}
			\end{block}
		\end{column}
		\begin{column}{.5\textwidth}
			\begin{block}{}
				
				\includegraphics[width=1\linewidth]{images/depron_sec_log08-29-1615-04-00}
											
			\end{block}
		\end{column}
	\end{columns}

\end{frame}



%\section{Normalised Radiation dynamics}
%%------------------------------------------------
%\foreach \n in {213,...,298} {%
%	\begin{frame}{Counts in detector 1 day \n}
%		\includegraphics[height=7cm]{images/hist/norm_depron_lb3d_polar\n.png}
%	\end{frame}%
%}



%\section{Normalised Radiation dynamics}
%------------------------------------------------		
%\begin{frame}
%	\frametitle{\insertsection} 
%	\framesubtitle{\insertsubsection}
%	\begin{figure}
%		\centering
%		\includegraphics[width=0.7\linewidth]{images/Graph3}
%		\caption{}
%		\label{fig:graph3}
%	\end{figure}
%\end{frame}
%------------------------------------------------			
\section{Summary}

\begin{frame}
	\begin{center}					
		\frametitle{\insertsection}
		\begin{itemize}
			\item Depron works properly register streams of charged particles and accumulated dose, taking into account the spectrum of energy release.
			\item Neutron fluxes registered
			\item Absorbed Doses per day is about 
		\end{itemize}
	\end{center}
\end{frame}

%------------------------------------------------			
\section{Future plans}

\begin{frame}
	\begin{center}					
		\frametitle{\insertsection}
		
		\begin{itemize}
			\item Dynamics of doses per day with the division of the anomaly and the contribution of the auroral zone and GCR
			
			\item According to GCR receive the latitudinal distribution of the dose rate (in L)
			
			\item The study of intensity variations  in the zone of L $ ~ $ 4-6 			
			
		\end{itemize}
	\end{center}
\end{frame}
	
%------------------------------------------------		


%\bgroup

%\usebackgroundtemplate{%
%	\tikz[overlay,remember picture] \node[opacity=0.3, at=(current page.center)] {
%	%	\includegraphics[width=\paperwidth]{images/sborka_1_3D}};
%}


%	\egroup	

\begin{frame}
	
\begin{center}
		\LARGE {Thank you for Attention!}
\end{center}
	
	Key Personalies:
	\begin{itemize}
		\item Victor Benghin
		
		\item Ivan~Zolotarev
		
		\item Oleg~Nechayev
		
		\item Alexander~Amelyushkin
		
		\item Nikolai~Vedenkin
		
		\item Vasiliy~Petrov
		
		\item M.I.~Panasyuk
		
		\item I.V.~Yashin
		
	\end{itemize}
	
	Contact email:
	brilkov@yandex.ru 
\end{frame}
			

	
