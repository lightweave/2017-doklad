\documentclass[]{article}
\usepackage{lmodern}
\usepackage{amssymb,amsmath}
\usepackage{ifxetex,ifluatex}
\usepackage{fixltx2e} % provides \textsubscript
\ifnum 0\ifxetex 1\fi\ifluatex 1\fi=0 % if pdftex
  \usepackage[T1]{fontenc}
  \usepackage[utf8]{inputenc}
\else % if luatex or xelatex
  \ifxetex
    \usepackage{mathspec}
  \else
    \usepackage{fontspec}
  \fi
  \defaultfontfeatures{Ligatures=TeX,Scale=MatchLowercase}
\fi
% use upquote if available, for straight quotes in verbatim environments
\IfFileExists{upquote.sty}{\usepackage{upquote}}{}
% use microtype if available
\IfFileExists{microtype.sty}{%
\usepackage[]{microtype}
\UseMicrotypeSet[protrusion]{basicmath} % disable protrusion for tt fonts
}{}
\PassOptionsToPackage{hyphens}{url} % url is loaded by hyperref
\usepackage[unicode=true]{hyperref}
\hypersetup{
            pdfborder={0 0 0},
            breaklinks=true}
\urlstyle{same}  % don't use monospace font for urls
\IfFileExists{parskip.sty}{%
\usepackage{parskip}
}{% else
\setlength{\parindent}{0pt}
\setlength{\parskip}{6pt plus 2pt minus 1pt}
}
\setlength{\emergencystretch}{3em}  % prevent overfull lines
\providecommand{\tightlist}{%
  \setlength{\itemsep}{0pt}\setlength{\parskip}{0pt}}
\setcounter{secnumdepth}{0}
% Redefines (sub)paragraphs to behave more like sections
\ifx\paragraph\undefined\else
\let\oldparagraph\paragraph
\renewcommand{\paragraph}[1]{\oldparagraph{#1}\mbox{}}
\fi
\ifx\subparagraph\undefined\else
\let\oldsubparagraph\subparagraph
\renewcommand{\subparagraph}[1]{\oldsubparagraph{#1}\mbox{}}
\fi

% set default figure placement to htbp
\makeatletter
\def\fps@figure{htbp}
\makeatother


\date{}

\begin{document}

\section{Всплески интенсивности}\label{header-n0}

Надо заменить картинку широтного распределения на временную серию,
которая показывает и возрастание и прохождение отрогов пояса или
аномалии

\begin{center}\rule{0.5\linewidth}{\linethickness}\end{center}

\section{История исследования}\label{header-n6}

Список характерных публикаций по теме возрастаний потоков частиц в
высокогоширотных областях.

\begin{itemize}
\item
  статья 1962
\item
  статья 2014
\item
  статья 2016
\end{itemize}

Новизна нашего исследования заключается в оценке дозиметрических
характеристик всплесков.

\begin{center}\rule{0.5\linewidth}{\linethickness}\end{center}

\section{Кратко по истории вопроса}\label{header-n22}

Если кто то из коллег осведомлен о публикациях дозиметрических
характеристик описанных всплесков, мы будем очень благодарны за указание
таких работ.

\begin{center}\rule{0.5\linewidth}{\linethickness}\end{center}

\section{План доклада}\label{header-n26}

\begin{enumerate}
\def\labelenumi{\arabic{enumi}.}
\item
  описание ДЭПРОН, коэффициенты перехода от внутренних едениц к потоку и
  дозе. Схема расположения детекторов прибора и защиты около детекторов,
  минимальные энергии проникающих частиц.
\item
  особенности алгоритма обработки данных для поиска всплесков
\item
  доступность данных и порядок наземной обработки
\item
  результаты без всплесков, здесь график рассеяния для аномалии и
  полярной области. Скаттерплот: счёт счет нижнего детектора от счета
  верхнего детектора. Ещё по дозе?
\item
  статистика всплесков и их феноменология. Критерии отбора событий.
\item
  статистика всплесков и географические распределения
\item
  связь с параметрами солнечной активности
\item
  дозиметрические характеристики всплесков
\end{enumerate}

\begin{center}\rule{0.5\linewidth}{\linethickness}\end{center}

\section{Заключение}\label{header-n53}

\end{document}
